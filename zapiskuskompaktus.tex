\documentclass[a3paper,10pt]{extarticle}
% \usepackage[utf8]{inputenc}
\usepackage[mathletters]{ucs}
\usepackage[utf8x]{inputenc}

\usepackage{fancyhdr}

\usepackage[pdftex]{graphicx} % Required for including pictures
\usepackage[pdftex,linkcolor=black,pdfborder={0 0 0}]{hyperref} % Format links for pdf
\usepackage{calc} % To reset the counter in the document after title page
\usepackage{enumitem} % Includes lists

\usepackage{textcomp}
\usepackage{eurosym}

\usepackage{ dsfont } % font za množice
% tabele
\usepackage{array}
\usepackage{wrapfig}

\usepackage{tikz,forest}
\usetikzlibrary{arrows.meta}

\frenchspacing % No double spacing between sentences
\setlength{\parindent}{0pt}
\setlength{\parskip}{0.1em}

\usepackage{mathtools}
\usepackage{blkarray, bigstrut} %


\usepackage{amssymb,amsmath,amsthm,amsfonts}
\usepackage{multicol,multirow}
\usepackage{calc}
\usepackage{ifthen}
\usepackage{tabularx}
\usepackage[landscape]{geometry}
\usepackage{listings}
\usepackage{inconsolata}
%\usepackage[colorlinks=true,citecolor=blue,linkcolor=blue]{hyperref}
%\usepackage{accents}
\usepackage{pdfpages}

\newcommand{\vect}[1]{\accentset{\rightharpoonup}{#1}}

\ifthenelse{\lengthtest { \paperwidth = 11in}}
    { \geometry{top=.5in,left=.5in,right=.5in,bottom=.5in} }
	{\ifthenelse{ \lengthtest{ \paperwidth = 297mm}}
		{\geometry{top=1cm,left=1cm,right=1cm,bottom=1cm} }
		{\geometry{top=1cm,left=1cm,right=1cm,bottom=1cm} }
	}
\pagestyle{empty}
\makeatletter
\renewcommand{\section}{\@startsection{section}{1}{0mm}%
                                {-1ex plus -.5ex minus -.2ex}%
                                {0.5ex plus .2ex}%x
                                {\normalfont\large\bfseries}}
\renewcommand{\subsection}{\@startsection{subsection}{2}{0mm}%
                                {-1explus -.5ex minus -.2ex}%
                                {0.5ex plus .2ex}%
                                {\normalfont\normalsize\bfseries}}
\renewcommand{\subsubsection}{\@startsection{subsubsection}{3}{0mm}%
                                {-1ex plus -.5ex minus -.2ex}%
                                {1ex plus .2ex}%
                                {\normalfont\small\bfseries}}
\makeatother
\setcounter{secnumdepth}{0}
%\setlength{\parindent}{0pt}
%\setlength{\parskip}{0pt plus 0.5ex}

% listings okolje za psevdo kodo
\lstnewenvironment{koda}[1][] %defines the algorithm listing environment
{   
    \lstset{ %this is the stype
        mathescape=true,
        basicstyle=\scriptsize, 
		columns=flexible,
        keywordstyle=\bfseries\em,
        keywords={,vhod, izhod, zacetek, konec, koncamo, ponavljaj, dokler, ce, vrni, za, vsak, vse, v, sicer,} %add the keywords you want, or load a language as Rubens explains in his comment above.
        xleftmargin=.1\textwidth,
		tabsize=4,
		%frame=leftline,xleftmargin=5pt,xrightmargin=5pt,framesep=5pt,
		%inputencoding = utf8,
		extendedchars = true,
		literate={ž}{{\ˇz}}1 {š}{{\ˇs}}1 {č}{{\ˇc}}1 {Ž}{{\ˇZ}}1 {Š}{{\ˇS}}1 {Č}{{\ˇC}}1,
        #1 % this is to add specific settings to an usage of this environment (for instnce, the caption and referable label)
    }
}
{}
% -----------------------------------------------------------------------
\begin{document} 

\begin{multicols}{4}
\setlength{\premulticols}{1pt}
\setlength{\postmulticols}{1pt}
\setlength{\multicolsep}{1pt}
\setlength{\columnsep}{2pt}

\section{TODO}
\begin{itemize}
	\item Znani problemi (max/min cut, perfect matching, quicksort, ...)
\end{itemize}

\section{Uporabne formule}
\[ H_n = \sum_{k=1}^n \frac{1}{k}\; \leq \; 1 + O(\log n)\]

\[
    \begin{aligned}
        \sum_{n=0}^{\infty} q^n &= \frac{1}{1-q} &
        \sum_{n=0}^{b} q^n &= \frac{1-q^{b+1}}{1-q}
        \\
        \sum_{n=a}^{\infty} q^n &= \frac{q^{a}}{1-q} &
        \sum_{n=a}^{b} q^n &= \frac{q^a-q^{b+1}}{1-q}
    \end{aligned}
\]

\[
    a^n - b^n = (a-b)(a^{n-1} + a^{n-2}b + ... + ab^{n-2} + b^{n-1})  
\]
\[ (x+y)^n = \sum_{k=0}^{n} \binom{n}{k} x^{n-k}y^{k} \]
\[ \frac{1}{(1-x)^n} = \sum_{k=0}^{n} \binom{n+k-1}{k} x^{k} \]

\[\binom{n}{k} = \frac{n^{\underline{k}}}{k!} = \frac{n!}{k!(n-k)!} = \binom{n}{n-k}\]

\subsubsection{Izbori}
Imamo $n$ oštevilčenih kroglic. Na koliko načinov lahko izberemo $k$ kroglic?

\begin{center}
    \begin{tabular}{ m{6em} | c | c | } 
         & \textbf{s pon.} & \textbf{brez pon.}\\ 
        \hline
        \textbf{variacije} \emph{vrstni red je pomemben} & $n^k$ & $n^{\underline{k}}$ \\ 
        \hline
        \textbf{kombinacije} \emph{vrstni red ni pomemben} & $\binom{n+k-1}{k}$ & $\binom{n}{k}$ \\ 
    \end{tabular}
\end{center}

\section{Verjetnostni algoritmi za odločitvene probleme}
Odgovarjamo na vprašanje $\omega \in \Pi$? \\
\textbf{Las Vegas} algoritmi vedno vrnejo pravilen odgovor \\
\textbf{Monte Carlo} algoritmi lahko vrnejo napačen odgovor
\begin{itemize}
	\item tip 1: $P(\text{yes}\ |\ \omega \in \Pi) \geq \frac{1}{2}$ $P(\text{yes}\ |\ \omega \notin \Pi) = 0$ 
	\item tip 2: $P(\text{yes}\ |\ \omega \in \Pi) = 1$ $P(\text{yes}\ |\ \omega \notin \Pi) \leq \frac{1}{2}$
	\item tip 3: $P(\text{yes}\ |\ \omega \in \Pi) \geq \frac{3}{4}$ $P(\text{yes}\ |\ \omega \notin \Pi) \leq \frac{1}{4}$
\end{itemize}

\subsubsection*{Razredi kompleksnosti odločitvenih problemov}
\begin{itemize}
	\item RP (randomized polynomial time): \\
	$\exists$ Monte Carlo tipa 1, ki v najslabšem primeru deluje v polinomskem času.
	\item co-RP: \\ 
	$\exists$ Monte Carlo tipa 2, ki v najslabšem primeru deluje v polinomskem času.
	\item BPP (bounded-error probabilistic polynomial time): $\exists$ Monte Carlo tipa 3, ki v najslabšem primeru deluje v polinomskem času.
	\item ZPP (zero-error probabilistic polynomial time): \\ 
	$\exists$ Las Vegas algoritem, ki deluje v pričakovanem polinomskem času. \\
	Ali (ekvivalentna definicija): $\exists$ alg, ki v najslabšem primeru deluje v polinomskem času in vedno vrne pravilen odgovor ali "ne vem" in $P(\text{"ne vem"}) < \frac{1}{2}$.
\end{itemize}

$\text{ZPP} = \text{RP} \cap \text{co-RP}$,\ \ $\text{P} \subset \text{ZPP}$,\ \ $\text{RP} \cup \text{co-RP} \subset \text{BPP}$

\section{Neenakost Chernoffa}
$X_1, \dots, X_n$ neodvisne slučajne spremenljivke, $X_i \in \{0, 1\}$, $X = \sum_{i=1}^n X_i$, $\mu = E(X)$. Potem za vsak $\delta \in (0,1)$ velja:

\begin{align*}
	P(X - \mu \geq \delta \mu) &\leq e^{-\frac{\delta^2 \mu}{2+\delta}} \leq e^{-\frac{\delta^2 \mu}{3}} \\
	P(\mu - X \geq \delta \mu) &\leq e^{-\frac{\delta^2 \mu}{2}} \leq e^{-\frac{\delta^2 \mu}{3}} \\
	P(|X - \mu| \geq \delta \mu) &\leq 2e^{-\frac{\delta^2 \mu}{3}} \\
\end{align*}

\section{Verjetnostni algoritmi za aproksimacijo}
Verjetnostni algoritem izračuna $(\epsilon, \delta)$-aproksimacijo za $V$, če vrne $X$ tako, da velja:
\[ P(|X-V| \leq \epsilon V ) \geq 1 - \delta \]

Naj bodo $X_1, \dots X_m$ slučajne spremenljivke, $\mu = E(X_i)$, $Y = \frac{\sum X_i}{m}$. 
Če je $m \geq \frac{3\ln(2/\delta)}{\epsilon^2 \mu}$, potem velja:
\[ P(|X-\mu| \geq \epsilon \mu) \leq \delta\]

in $Y$ je $(\epsilon, \delta)$-aproksimacija za $\mu$.

\section{Polinomi}
Naj bo $\mathbb{F}$ polje. Stopnja polinoma $p \in \mathbb{F}[x_1, \dots, x_n]$ je $\deg(p(x_1, \dots, x_n)) = \deg(p(x, \dots, x))$

\subsubsection{Schwartz-Zippelov izrek}
Naj bo $p \in \mathbb{F}[x_1, \dots, x_n]$ in $\deg(p) = d \geq 0$. Naj bo $S \subseteq \mathbb{F}$ poljubna končna podmnožica. Za naključno izbiro (enakomerno) $r \in S^n$ velja:
\[ P(p(r) = 0) \leq \frac{d}{|S|} \]

\section{Verjetnost}
\textbf{Verjetnost} na $(\Omega, \mathcal{F})$ je preslikava $P: \mathcal{F} \to \mathbb{R}$ z lastnostmi:

\begin{itemize}
    \item $P(A) \geq 0$ za $\forall A \in \mathcal{F}$
    \item $P(\Omega) = 1$
    \item Za paroma nezdružljive (disjunktne) dogodke $\{ A_i \}_{i=1}^\infty $ velja \textit{števna aditivnost}
    \[ P(\bigcup_{i=1}^\infty A_i) = \sum_{i=1}^\infty P(A_i)\]
    \item $P(\emptyset) = 0$
    \item $P$ je končno aditivna.
    \item $P$ je \textit{monotona}: $A \subseteq B \implies P(A) \leq P(B)$
    \item $P(A^\complement) = 1 - P(A)$
    \item $P$ je \textit{zvezna}:
    \[ A_1 \subseteq A_2 \subseteq \dots \implies P\big(\bigcup_{i=1}^\infty\big) = \lim_{i \to \infty} P(A_i)\]
    \[ B_1 \supseteq B_2 \supseteq \dots \implies P\big(\bigcap_{i=1}^\infty\big) = \lim_{i \to \infty} P(B_i)\]
\end{itemize}

\subsection{Matematično upanje}
Za slučajno spremenljivko $X: \Omega  \to \mathbb{Z}$
\[ E(X) =  \sum_{c\in \mathbb{Z}} c P(X = c)\]

\subsubsection{Lastnosti}
\[ E(f(X)) = \sum_{c \in \mathbb{Z}} f(c) P(X = c) \]

\textit{Linearnost}: za poljubne sl. sprem $X_1, \dots, X_n$ velja:
\[ E(a_1 X_1 + \dots a_n X_n) = a_1 E(X_1) + \dots + a_n E(X_n) \]

Če ima $|X|$ mat. up., ga ima tudi $X$ in velja 
\[|E(X)| \leq E(|X|) \]

Če obstaja $E(X^2)$ in $E(Y^2)$, obstaja tudi $E(XY)$ in velja:
\[|E(XY)| \leq E(|XY|) \leq \sqrt{E(X^2)E(Y^2)} \]

\subsection{Disperzija (varianca)}
\[D(X) = E((X - E(X))^2) = E(X^2) - (E(X))^2\]
Lastnosti: 
\begin{itemize}
    \item $D(X) \geq 0$
    \item $D(X) = 0 \iff P(X = E(X)) = 1$
    \item $D(aX) = a^2 D(X)$
\end{itemize}

Standardna diviacija/odklon:
\[ \sigma(X) = \sqrt{D(X)} \]
zanjo velja $\sigma (aX) = |a|\sigma(X)$.

\subsection{Neodvisnost}
Diskretno porazdeljeni sl. sprem. $X$ in $Y$ sta noedvisni, če velja:
\[ P(X = x_i, Y = y_j) = P(X = x_i)P(Y = y_j)\]
za vse $i, j$.

\subsection{Nekoreliranost}
Sl. sprem. $X$ in $Y$ sta nekorelirani, če velja:
\[ E(XY) = E(X)E(Y) \]
\[ X, Y \text{ neodvisni } \implies X,Y \text{ nekorelirani }\]

Če imata $X$ in $Y$, je nekoreliranost ekvivalentna zvezi:
\[ D(X+Y) = D(X) + D(Y)\]


\subsection{Neenakost Markova}
Če je $X$ ne negativna sl. sprem. z mat. up., potem je
\[P(|X| \geq a) \leq \frac{E(|X|)}{a} \quad \forall a > 0\]

\subsection{Neenakost Čebiševa}
Če ima $X$ disperzijo, je
\[ P(|X - E(X)| \geq a \sigma(X)) \leq \frac{1}{a^2}  \quad \forall a > 0\]
oziroma za $\varepsilon := a \sigma(X)$
\[ P(|X-E(X)| \geq \varepsilon) \leq \frac{D(X)}{\varepsilon^2}\]
\end{multicols}

\includepdf[pages={1}]{Porazd.pdf}
\end{document}